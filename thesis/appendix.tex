\appendix
\chapter{Full Application Examples}

The following sections provide a demonstration of working code using the current MELON library implementation in JRuby.

\section{News Server \& Reader Applications}

\begin{lstlisting}[caption={News Reader}, label={code:newsserverfull}]
require "melon"

# Initialize MELON
melon = Melon.with_zmq

# Initialize potential topics
topics = ["Politics", "Sports", "Business", "Technology", "Local"]

i = 0

loop do
  # Choose a topic
  topic = topics.sample
  
  # Generate headline for topic
  message = [topic, "#{topic} news item #{i+=1}"]
  
  # Write message
  melon.write message
  
  # Pause for a few seconds
  sleep rand(5)
end
\end{lstlisting}


\begin{lstlisting}[caption={News Reader}, label={code:newsreaderfull}]
require "melon"

unless ARGV[2]
  abort "news_reader.rb ADDRESS PORT TOPIC"
end

address = ARGV[0]
port = ARGV[1]
topic = ARGV[2]

# Initialize MELON
melon = Melon.with_zmq

# Add address of a news server
melon.add_remote port, address

# Set the template to retrieve a headline for the
# given topic
template = [topic, String]

# Read all relevant messages as they become available
loop do
  puts melon.read_all template
end
\end{lstlisting}

\section{Chat Application}

This example splits the implementation of a chat application into a library which provides most of the functionality and a small script to set up the environment for the user.

\begin{lstlisting}[caption={Chat Library}, label={code:chatlibfull}]
# Encapsulate chat communication
class Chat
  def initialize name
    @name = name
    @melon = Melon.with_zmq
  end

  def add_remote port
    @melon.add_remote port
  end
  
  # Write out a chat message
  def chat message
    @melon.write [@name, message]
  end

  # Get all unseen messages
  def read_messages
    @melon.read_all [String, String]
  end
 
  # Print out all messages except our own  
  def print_messages messages
    messages.each do |name, message|
      unless name == @name
        puts "\n<#{name}> #{message}"
      end
    end
  end

  # Read and show messages in a separate thread
  def monitor
    Thread.new do
      loop do
        print_messages read_messages
      end
    end
  end

  # Main loop for chatting
  def start
    monitor

    # Send chat messages from user
    loop do
      print "? "
      message = gets.strip

      unless message.empty?
        chat message
      end
    end
  end
end
\end{lstlisting}

The chat client in Listing \ref{code:chatclientfull} is a simple script to set up a chat client that talks to other local chat clients. This makes it simple to try on a single machine.

\begin{lstlisting}[caption={Local Chat Client}, label={code:chatclientfull}]
require "melon"
require "chat"

# Get user's name
print "Name: "
name = gets.strip

# Initialize chat library
chat = Chat.new name

# Add processes running on the same machine but different ports
loop do
  print "Remote port: "
  chat.add_remote gets.strip.to_i

  print "Add another (y/n)? "
  break unless gets.downcase.start_with? "y"
end

# Start chatting
chat.start
\end{lstlisting}