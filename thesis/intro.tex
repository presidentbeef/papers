\chapter{Introduction}

\section{Motivation}

Mobile ad hoc networks (MANET) comprised of small, mobile, wireless devices present a new and challenging area for the development and deployment of applications. As consumer devices equipped with WiFi capabilities such as smartphones become more widespread, the possibility of impromptu networks also increases. Mobile applications are no longer limited to stand-alone or client-server programs, but can interact and form useful networks directly with each other. Such networks are ideal for situations in which there is no time to set up a fixed access point, or when there is no fixed infrastructure available. Many new applications, particularly in the consumer space, are being applied to MANET, including collaborative software such as shared whiteboards, impromptu networks for communication and entertainment, and peer-to-peer applications for file sharing. While typical examples of MANETs include military units or disaster recovery scenarios, MANETs are also useful inside buildings where cellular reception might be unavailable, where censorship blocks free speech, or even for instant networked gaming between nearby friends.

In mobile ad hoc networks (MANET), high nodal mobility causes frequent topology and route changes, making it difficult to maintain network connections between nodes. Many routes in the network span multiple wireless hops and may experience dramatic and unexpected fluctuations in quality. The combination of mobility and wireless communication creates highly dynamic network topologies in which frequent, possibly permanent disconnections are commonplace, rather than exceptional events. The dynamics of the network and the wireless channel requires changes to the networking stack and alternative solutions at the application level.

While there has been a large amount of work focused on the network stack for MANETs, especially routing, the application layer is not insulated from the challenges faced at the networking layers. Mobile applications face several challenges when compared with programs intended for standard desktops: mobile devices are generally constrained in many ways: the screen size, processor power, memory, and battery power are often limited. Development platforms for mobile devices typically provide basic libraries for application support such as menus and access to data stored on the device. Networking, however, is generally limited to sockets, TCP/IP, and HTTP. In particular, applications are expected to either be stand-alone, like a calculator, or to only be using the network in a client-server manner, such as accessing websites or email servers.

However, most laptop computers are already equipped with WiFi which can operate in ad-hoc mode. Smartphones with WiFi are becoming ubiquitous. In 2013, surveys showed 91\% of adults in the United States had cell phones, and 56\% of those are smartphones\cite{cellphones}. Among teens, 78\% had a cell phone, of which 37\% are a smartphone\cite{teenphones}. Add smartphones to the proliferation of tablets and laptops and the ability for consumers to form mobile ad hoc networks (MANETs) is quickly becoming possible. However, applications designed for these networks remain in short supply.

One way to encourage creation of MANET applications is to simplify communication between devices. Given the challenges of distributed communication in such volatile networks, MANET applications often implement an abstraction layer for network communication. The majority of these abstraction layers are based on traditional distributed computing paradigms which were not designed for unreliable, rapidly-changing wireless networks. We have examined these paradigms and the performance of their implementation for MANETs and found them to be unsuitable for general purpose communication needs of MANET applications. Therefore, we have designed a new communication paradigm specifically to meet the challenges of MANETs, rather than modify an existing paradigm which was not originally intended for the MANET environment.

\section{Contributions}

The first contributions of this dissertation are two comparative studies of communication paradigms used in MANET applications. The first study is a survey of existing libraries and languages used to support MANET applications. Included in the survey is a quantitative comparison using a subset of the surveyed projects to examine how well they perform in a realistic MANET environment. This was the first such quantitative comparison of these projects in identical scenarios.

Once we had surveyed existing projects, it became clear the majority of projects rely on three traditional distributed computing paradigms: publish/subscribe, remote procedure calls, and tuple spaces. In the previous study, we compared performance of projects, but the implementations were in different languages with different levels of completeness and rigor. This made it difficult to conclude anything regarding the underlying communication paradigms. To directly compare the paradigms themselves and determine their suitability as the basis for MANET applications, we implemented canonical versions of the paradigms with as much shared code as possible. In the second contribution of this dissertation, we investigated the impact of wireless and mobility at the application layer for the different paradigms via an emulated network stack, detailed wireless models, and real applications.

After studying the traditional communication paradigms and their adaptations to the MANET context, we designed a new paradigm, MELON, to specifically to meet the challenges of distributed communication in MANETs. MELON provides message persistence, reliable FIFO multicast, read-only messages, simple message streaming, private messages, and efficient bulk operations. To operate well in the distributed and unreliable MANET environment, the design of MELON avoids any global state or locking and performs all operations on-demand. Thus, the third contribution in this dissertation is the design of the MELON communication language.

Our fourth contribution is a prototype implementation of MELON. In order to evaluate the practicality of using MELON in MANET applications, we implemented a prototype of MELON as well as implementing better versions of the three traditional paradigms from above. Again, the four paradigms share as much code as possible in order to eliminate performance differences caused by different implementations. We then compared the performance of MELON with the three traditional paradigms. Besides the quantitative comparison, we also implemented several example applications using MELON to demonstrate its utility in a number of scenarios.

Finally, while empirically evaluating all these paradigms, we created an experiment coordination framework implemented using MELON itself. The framework is responsible for managing the network emulator, running the applications under review, collecting output from the applications, and collating the results. The framework coordinates the emulator and applications to start and stop at the same time, as well as communicating experiment parameters to the applications. This experiment coordination framework is the final contribution of this dissertation.

\section{Dissertation Organization}

The remainder of this dissertation is laid out as follows:

Chapter \ref{chapter:paradigms} reviews the challenges presented by MANETs and the fundamental communication paradigms which have been applied to MANET applications. In Chapter \ref{chapter:existing} we compare existing projects providing communication libraries for MANETs as well as canonical implementations of the underlying communication paradigms. We present both qualitative and quantitative analysis of the projects and paradigms. Chapter \ref{chapter:model} presents the design of MELON and its communication model, then covers the prototype implementation of MELON in Chapter \ref{chapter:implementation}.

Several case studes are presented in Chapter \ref{chapter:cases} using MELON in applications, including the experiment coordination framework used in our later evaluations. Chapter \ref{chapter:evaluation} includes performance comparisons between MELON and traditional communication paradigms and discusses the suitability of MELON as a basis for MANET applications. Finally, our results and future work are presented in Chapter \ref{chapter:conclusions}. Full MELON application code for a news server/reader and a chat room example may be found in Appendix A.