\chapter{Conclusions}\label{chapter:conclusions}

In this dissertation, we have presented an investigation into communication paradigms for applications operating in mobile ad hoc networks. Starting with an qualitative survey of existing projects to support communication MANETs, we then compared performance of representative projects using real applications in an emulated MANET environment. We showed the wireless and mobile environment is dramatically different from a static wired network and communication libraries must be adapted to the MANET environment. We determined \textit{disconnection handling}, \textit{resource addressing and discovery}, and \textit{flexible communication} to be important issues to address for MANET communication libraries.

We found most projects were based on traditional distributed computing paradigms, but it was not possible to assert any conclusions regarding the underlying paradigms, since the project implementations compared used different languages and were of varying quality. In order to study the paradigms' performance in a quantitative manner, we implemented our own versions of three commonly-used paradigms (publish/subscribe, RPC, and tuple spaces) with as much shared code as possible. This allowed us to fairly compare the paradigms using real applications.

After empirically investigating the traditional paradigms, we found message persistence and connectionless operations to be especially beneficial for MANET applications. RPC relies too heavily on synchronous communication between hosts and is not well-suited for group communication with an unknown set of participants or unreliable remote hosts, although it does provide reliable unicast communication. Publish/subscribe is fast and light, but does not provide reliable delivery or convenient unicast communication. Implementing a system of brokers to provide persistence and manage subscriptions is complex in a MANET. Tuple spaces, not even originally designed for distributed systems, do provide message persistence and flexible communication, but are hampered by strict semantics and lack of support for message streams. None of the three paradigms provided private unicast communication, a common requirement for modern mobile applications.

While adapting existing research and solutions to new problem domains is a valid first step, it became clear in this research that traditional distributed computing paradigms, were not designed for nor suited to the challenges of MANETs. Therefore, we proposed a new communication model for MANETs called MELON. MELON provides message persistence, basic access controls, on-demand operations, bulk message retrieval, FIFO message ordering, and support for simple message streaming. We believe MELON is well-suited to supporting MELON applications while offering a simple and easily implemented set of operations. 

We have examined several case studies using MELON to implement applications and found it well-suited to provide convenient and efficient communication for MANET applications. To evaluate MELON, we again implemented the traditional paradigms in order to share a common codebase with MELON and provide a fair comparison. As part of the evaluation, we implemented an experiment coordination framework implemented using MELON itself. In our experiments, the prototype implementation of MELON performed well in comparison to the existing paradigms, suggesting it is a viable approach to MANET communication.

\section{Future Work}

This dissertation presents only the initial design of a MANET communication paradigm. Although the prototype implementation performed well, it is implemented naively. In particular, it does not attempt to leverage multicast communication, which may improve performance. Very little investigation has been performed in determining appropriate policies for message replication and garbage collection.

Security is a growing concern, especially for wireless communication. In this dissertation, we defined private communication as that which cannot be accessed by unauthorized nodes from within the communication paradigm. However, this ignores the reality of how easily it is to eavesdrop on wireless communication. Naturally encrypted connections are desired, but verifying identity of nodes in a decentralized network is a challenge. Relatedly, while MELON includes a proposal for ``directed messages" which can only be read by their addressee, we have not provided any mechanism for assigning and verifying identities.

As mobile devices become more accessible and open, we hope to see MELON adapted to work on consumer mobile devices such as smartphones to enable the development of more decentralized applications working together in mobile ad hoc networks.